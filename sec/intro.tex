\section{Introduction}

\pagenumbering{arabic}

This project entitled {\bf\em Departmental Store Management System} is a web based
application that manages various activities done in department store.
These activities include ordering goods/products from suppliers/producers,
entering the new arrival of products into the system, managing those products
in the department store and making bills while selling those products to customers.
The system can store information of the products in a very structured way.
Bills are created and details of each transaction can be viewed whenever required.
The interface of the system will be user friendly and as simple as possible
yet powerful requiring little training of operating staff.

Our system stores the detail of supplier, product, customer bills, department
store staff as main entities. The department store manager regularly checks the 
product in stock and orders those that are out of stock. Goods and product come 
from many suppliers. Each supplier supply one or more products but one product 
is supplied by a unique supplier. These products are then classified into several 
categories and arranged categorically in separate racks. Details of every products 
(like in which category does it belong to, where it is located, available number, 
price) etc. are recorded and maintained in central database. An employee is assigned 
to enter the necessary information about the product into the database system of 
the store. A customer uses the system in order to find the product he is searching 
for and he gathers all the products he wishes to buy and brings it at the payment 
section of the store. An employee then enters the product codes of the customer 
purchase into the system and bill/invoice is prepared by the system. A customer 
makes the payment. Each transaction is stored in database for future references. 
So this is all about the workflow of our system.

As already mentioned above, this is a web based application system. A web-based 
application is any application that uses a website as the interface
(the ``front-end''). Users access the application from any computer connected
to the Internet using a standard browser, instead of using an application 
that has been installed on their local computer. Almost any desktop software 
can be developed as a web-based application. With web-based applications, 
users access the system via a uniform environment---the web browser. While the 
user interaction with the application needs to be thoroughly tested on different 
web browsers, the application itself needs only be developed for a single
operating system. There's no need to develop and test it on all possible 
operating system versions and configurations. This makes development and 
troubleshooting much easier.
