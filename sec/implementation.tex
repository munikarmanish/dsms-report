\section{Implementation Technology}

\subsection{Django}

Django is a widely-used Python web application framework. It is a
well-established open source project with tens of thousands of users and
contributors spread across the planet. It is simply a collection of libraries
written in the Python programming language. We have used Django version 1.10 in
our current project implementation.

\subsubsection{Why Django?}

Django is a free and open source web application framework,
written in Python.It helps us to develop websites faster and easier. It takes
care of user authentication, content administration, sitemaps, RSS feeds, and
many more tasks right out of the box.  Its user authentication system
provides a secure way to manage user accounts and passwords preventing from SQL
injection cross-site scripting, cross-site request forgery and clickjacking. It
is exceedingly scalable and incredibly versatile.

\subsection{MySQL}

MySQL is an open source relational database management
system (RDBMS) based on Structured Query Language (SQL). It is the world's most
popular open source database. It is developed, marketed, and supported by MySQL
AB, which is a Swedish company.

\subsubsection{Why MySQL?}

MySQL is most often associated with web-based
applications and online publishing. It supports large databases, up to 50
million rows or more in a table. It runs on many operating systems and
support several development interfaces. It is easy to use, scalable, fast and
secure.

\subsection{Where and How Were They Used?}

Since our project is mainly based on
managing the products and their transaction using web based application,
database is most important part of our project. Mysql database is used as
the database engine of our system. The system consists of 11 major tables
for different entities of our project, some of them are: Bill, Staff,
Product, Bill detail, Order, Supplier, etc. Django is connected with the
MySQL and models are created for each tables describing field details.

